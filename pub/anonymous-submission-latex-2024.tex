\documentclass[letterpaper]{article} % DO NOT CHANGE THIS
\usepackage[submission]{aaai24}  % DO NOT CHANGE THIS
\usepackage{times}  % DO NOT CHANGE THIS
\usepackage{helvet}  % DO NOT CHANGE THIS
\usepackage{courier}  % DO NOT CHANGE THIS
\usepackage[hyphens]{url}  % DO NOT CHANGE THIS
\usepackage{graphicx} % DO NOT CHANGE THIS
\urlstyle{rm} % DO NOT CHANGE THIS
\def\UrlFont{\rm}  % DO NOT CHANGE THIS
\usepackage{natbib}  % DO NOT CHANGE THIS AND DO NOT ADD ANY OPTIONS TO IT
\usepackage{caption} % DO NOT CHANGE THIS AND DO NOT ADD ANY OPTIONS TO IT
\frenchspacing  % DO NOT CHANGE THIS
\setlength{\pdfpagewidth}{8.5in} % DO NOT CHANGE THIS
\setlength{\pdfpageheight}{11in} % DO NOT CHANGE THIS
%
% These are recommended to typeset algorithms but not required. See the subsubsection on algorithms. Remove them if you don't have algorithms in your paper.
\usepackage{algorithm}
\usepackage{algorithmic}

%
% These are are recommended to typeset listings but not required. See the subsubsection on listing. Remove this block if you don't have listings in your paper.
\usepackage{newfloat}
\usepackage{listings}

% User imported packages
\usepackage{multirow}
\usepackage{pifont}
\usepackage{booktabs}
\usepackage{subcaption}
\usepackage{bm}
\usepackage{amsmath}
\usepackage{amssymb}
\usepackage{cleveref}
\usepackage{todonotes}
\usepackage{varwidth}
\usepackage{tikz}
\usepackage{graphicx}
\newcommand{\cmark}{\ding{51}} % checkmark symbol
\newcommand{\xmark}{\ding{55}} % X-mark symbol

\DeclareCaptionStyle{ruled}{labelfont=normalfont,labelsep=colon,strut=off} % DO NOT CHANGE THIS
\lstset{%
	basicstyle={\footnotesize\ttfamily},% footnotesize acceptable for monospace
	numbers=left,numberstyle=\footnotesize,xleftmargin=2em,% show line numbers, remove this entire line if you don't want the numbers.
	aboveskip=0pt,belowskip=0pt,%
	showstringspaces=false,tabsize=2,breaklines=true}
\floatstyle{ruled}
\newfloat{listing}{tb}{lst}{}
\floatname{listing}{Listing}
%
% Keep the \pdfinfo as shown here. There's no need
% for you to add the /Title and /Author tags.
\pdfinfo{
/TemplateVersion (2024.1)
}

% DISALLOWED PACKAGES
% \usepackage{authblk} -- This package is specifically forbidden
% \usepackage{balance} -- This package is specifically forbidden
% \usepackage{color (if used in text)
% \usepackage{CJK} -- This package is specifically forbidden
% \usepackage{float} -- This package is specifically forbidden
% \usepackage{flushend} -- This package is specifically forbidden
% \usepackage{fontenc} -- This package is specifically forbidden
% \usepackage{fullpage} -- This package is specifically forbidden
% \usepackage{geometry} -- This package is specifically forbidden
% \usepackage{grffile} -- This package is specifically forbidden
% \usepackage{hyperref} -- This package is specifically forbidden
% \usepackage{navigator} -- This package is specifically forbidden
% (or any other package that embeds links such as navigator or hyperref)
% \indentfirst} -- This package is specifically forbidden
% \layout} -- This package is specifically forbidden
% \multicol} -- This package is specifically forbidden
% \nameref} -- This package is specifically forbidden
% \usepackage{savetrees} -- This package is specifically forbidden
% \usepackage{setspace} -- This package is specifically forbidden
% \usepackage{stfloats} -- This package is specifically forbidden
% \usepackage{tabu} -- This package is specifically forbidden
% \usepackage{titlesec} -- This package is specifically forbidden
% \usepackage{tocbibind} -- This package is specifically forbidden
% \usepackage{ulem} -- This package is specifically forbidden
% \usepackage{wrapfig} -- This package is specifically forbidden
% DISALLOWED COMMANDS
% \nocopyright -- Your paper will not be published if you use this command
% \addtolength -- This command may not be used
% \balance -- This command may not be used
% \baselinestretch -- Your paper will not be published if you use this command
% \clearpage -- No page breaks of any kind may be used for the final version of your paper
% \columnsep -- This command may not be used
% \newpage -- No page breaks of any kind may be used for the final version of your paper
% \pagebreak -- No page breaks of any kind may be used for the final version of your paperr
% \pagestyle -- This command may not be used
% \tiny -- This is not an acceptable font size.
% \vspace{- -- No negative value may be used in proximity of a caption, figure, table, section, subsection, subsubsection, or reference
% \vskip{- -- No negative value may be used to alter spacing above or below a caption, figure, table, section, subsection, subsubsection, or reference

\setcounter{secnumdepth}{0} %May be changed to 1 or 2 if section numbers are desired.

% The file aaai24.sty is the style file for AAAI Press
% proceedings, working notes, and technical reports.
%

% Title

% Your title must be in mixed case, not sentence case.
% That means all verbs (including short verbs like be, is, using,and go),
% nouns, adverbs, adjectives should be capitalized, including both words in hyphenated terms, while
% articles, conjunctions, and prepositions are lower case unless they
% directly follow a colon or long dash
\iffalse\title{AAAI Press Anonymous Submission\\Instructions for Authors Using \LaTeX{}}
\author{
    %Authors
    % All authors must be in the same font size and format.
    Written by AAAI Press Staff\textsuperscript{\rm 1}\thanks{With help from the AAAI Publications Committee.}\\
    AAAI Style Contributions by Pater Patel Schneider,
    Sunil Issar,\\
    J. Scott Penberthy,
    George Ferguson,
    Hans Guesgen,
    Francisco Cruz\equalcontrib,
    Marc Pujol-Gonzalez\equalcontrib}
\affiliations{
    %Afiliations
    \textsuperscript{\rm 1}Association for the Advancement of Artificial Intelligence\\
    % If you have multiple authors and multiple affiliations
    % use superscripts in text and roman font to identify them.
    % For example,

    % Sunil Issar\textsuperscript{\rm 2},
    % J. Scott Penberthy\textsuperscript{\rm 3},
    % George Ferguson\textsuperscript{\rm 4},
    % Hans Guesgen\textsuperscript{\rm 5}
    % Note that the comma should be placed after the superscript

    1900 Embarcadero Road, Suite 101\\
    Palo Alto, California 94303-3310 USA\\
    % email address must be in roman text type, not monospace or sans serif
    proceedings-questions@aaai.org
%
% See more examples next
}
\fi

%Example, Single Author, ->> remove \iffalse,\fi and place them surrounding AAAI title to use it
\iffalse\title{My Publication Title --- Single Author}
\author{
    Author Name
}
\affiliations{
    Affiliation\\
    Affiliation Line 2\\
    name@example.com
}
\fi

%Example, Multiple Authors, ->> remove \iffalse,\fi and place them surrounding AAAI title to use it
\title{Learning Rate Optimization for Online Deep Learning}
\author{
    % Authors
    Lucas Cazzonelli\textsuperscript{\rm 1},
    Cedric Kulbach\textsuperscript{\rm 2},
}
\affiliations{
    % Affiliations
    \textsuperscript{\rm 1}FZI Research Center for Information Technology\\
    \textsuperscript{\rm 2}Affiliation 2\\
    cazzonelli@fzi.de, secondAuthor@affilation2.com
}


% REMOVE THIS: bibentry
% This is only needed to show inline citations in the guidelines document. You should not need it and can safely delete it.
\usepackage{bibentry}
% END REMOVE bibentry

\begin{document}

\maketitle


\begin{abstract}

	\noindent The efficient training via gradient-based optimization techniques is an essential building block of the success of deep learning.
	These techniques have been enabled in part by extensive research on the impact and the effective estimation of an appropriate learning rate.
	Despite the proliferation of data streams produced by IoT devices, digital platforms etc., previous research has been largely focused on batch learning, which assumes that all training data is available at once.
	However, because the gradual emergence and non-stationarity of data, as well as other characteristics, present additional challenges, the insights from batch learning may not be applicable to deep learning in streaming environments.
	In this work, we seek to address this knowledge gap by (i) evaluating and comparing common learning rate schedules and optimizers, (ii) exploring adaptations of these techniques and (iii) providing insights into effective learning rate tuning in the context of stream-based deep learning.


\end{abstract}

\section{Introduction}
Deep learning models have demonstrated exceptional performance in various domains, with the choice of optimizer playing a crucial role in achieving outstanding results.
In the context of batch learning, where all data is available at a time, extensive research has been conducted to explore different optimization techniques for deep learning architectures.
Numerous methods have emerged to effectively update the weights of these architectures
However, the investigation of optimizer choices in online learning, where models must adapt to evolving data streams, remains relatively limited.

This paper aims to bridge this knowledge gap by investigating how the choice of optimizer changes when transitioning from batch learning to online learning scenarios. Specifically, we address the following research questions:
\begin{itemize}
	\item How does the choice for the optimizer change from batch to online learning?
	\item What are practical choices for gradient-based online training of deep architectures in online learning?
	\item Are adaptive optimization methods better suited in Online Deep Learning?
\end{itemize}
\todo{Put LR tuning graphic here?}
\begin{figure}[ht]
	\centering
	\begin{tikzpicture}
		% Upper image
		\node[inner sep=0pt] (upper) {\includegraphics[width=0.4\textwidth]{figures/adam_trajectory_drift_reset1.pdf}};

		% Lower image
		\node[inner sep=0pt, below=3mm of upper] (lower){\includegraphics[width=0.4\textwidth]{figures/adam_trajectory_drift_reset2.pdf}};

		\path ([xshift=-8pt]upper) -- ([xshift=-8pt]lower) node[midway] (text){Concept Drift};
		\draw[->] ([xshift=-5pt, yshift=5pt]upper.south-|text.west) -- ([xshift=-5pt, yshift=-3pt]lower.north-|text.west);
		\draw[->] ([xshift=5pt, yshift=5pt]upper.south-|text.east) -- ([xshift=5pt, yshift=-3pt]lower.north-|text.east);

	\end{tikzpicture}
	\caption{Parameter trajectory of Adam~\cite{kingmaAdamMethodStochastic2017b} with or without adaptation to concept drift on synthetic data stream with abrupt concept drift. Marker colors depict the expected prequential loss over the last 16 data instances.}
\end{figure}
For the first research question, we explore how the selection of an optimizer differs when moving from the traditional batch learning setting to the dynamic online learning scenario. We examine the suitability of various optimizer choices in online learning and their impact on model performance.

The second research questions investigates practical choices for gradient-based online training of deep architectures.
We analyze different optimization techniques and explore their effectiveness in adapting to evolving data streams while maintaining model performance.
The third research question focuses on the performance of adaptive methods in online deep learning scenarios.
These methods dynamically adjust the learning rate based on gradient characteristics, allowing models to adapt more effectively to changing data patterns. We compare the performance of adaptive methods against other optimization approaches to determine their suitability for online deep learning tasks.
Through our in-depth analysis and experimentation, we aim to enhance our understanding of optimizer choices in online deep learning.
By shedding light on the impact of optimizers, learning rates, and batch sizes, and comparing the effectiveness of adaptive methods, we aim to enable researchers and practitioners to make informed decisions when selecting optimization techniques for real-time learning tasks.

\section{Learning Rate Scheduling}

In the following, we will briefly outline the most important differences between the influence of the learning rate of first order gradient-based optimization methods in streaming- and in conventional batch learning environments.

First order gradient-based optimization approaches like stochastic gradient descent and its derivatives aim to iteratively minimize the error of a DL model using only first order gradient information at each step $t$. We denote the gradient of the prediction error for a mini-batch of training samples $y_t, X_t \sim p_t$ with respect to model-parameters $\theta$ as
\begin{equation}
	g_t = \nabla_{\theta} \mathcal{L}(y_t, f(X_t; \theta)),
\end{equation}
where $\mathcal{L}$ represents a loss function.
Using this notation, the parameter values that SGD yields at timestep $t$ are given by
\begin{equation}
	\theta_{t}  = \theta_{0} - \sum_{i=0}^t \eta_t \cdot g_t,
\end{equation}
where $\eta_t$ denotes the learning rate. The task of optimizing the learning rate in a batch learning setting can then be defined as
\begin{equation}
	\label{eq:batch_lr_optim}
	\begin{alignedat}{3}
		& \!\min_{\eta_0, \ldots, \eta_T} & \, & \sum_{i=1}^{V}\mathcal{L}(y_i, f(X_i; \theta_T)) &    &                              \\
		& \text{s.t.}                     &    & X_i, y_i  \sim p^{(v)}                           & \  & \forall i \in {1, \ldots, V}
		,
	\end{alignedat}
\end{equation}
where $p^{(v)}$ is a distribution of held-out validation data and $T$ the total number of training steps.

The primary trade-off to consider with respect to the choice of $\eta$ is that increasing the learning rate speeds up convergence, but at the same time also increases stochasticity and therefore leads to the divergence of the training criterion beyond a certain threshold.~\cite{bengioPracticalRecommendationsGradientbased2012}.
In fact, \citet{smithBayesianPerspectiveGeneralization2018}, found that when modelling SGD as a stochastic differential equation, the “noise scale” is directly tied to $\eta$~\cite{smithBayesianPerspectiveGeneralization2018}.
In biological terms, increasing the learning rate increases plasticity, whereas decreasing it increases stability. % Maybe remove? I like the biological analogy, but there are no sources supporting the claim. 

To ensure fast convergence at the start of training, while mitigating jumping around potential minima at later stages it is common to use a decaying schedule starting with a large learning rate that decreases over time.
An additional benefit of this approach is that of potentially better generalization, since larger learning rates can help skipping over sharp minima with poor generalization~\cite{hochreiterFlatMinima1997,chaudhariEntropySGDBiasingGradient2017}.
Some have likened this procedure to simulated annealing, which shifts its focus from exploration at high temperatures to exploitation once temperatures have sufficiently decreased~\cite{smithDonDecayLearning2018}.

Commonly used forms of decay is exponential decay, where $\eta_{t}$ is calculated as
\begin{equation}
	\eta_{t} = \eta_0 * \gamma^t,
\end{equation}
with $\gamma < 1$, and stepwise decay, which can e.g. be defined as
\begin{equation}
	\eta_{t} = \eta_0 * \gamma^{\lfloor t/s \rfloor},
\end{equation}
with an interval between steps of $s$.

Other popular options include cyclic learning rate schedules which oscillate $\eta$ between a base- and a maximum or minimum value over a predefined interval.
For a basic triangular cycle, the learning rate calculates as
\begin{equation}
	\eta_t = \eta_0 + \frac{\hat{\eta} - \eta_0}{s} \cdot \min_{i\leq\lceil t/s \rceil} \{|t-i\cdot s|\}
\end{equation}
Some studies~\cite{smithCyclicalLearningRates2017, smithSuperConvergenceVeryFast2018a} have found cyclic schedules to significantly speed up the convergence of neural networks even when compared to adaptive techniques like Adam~\cite{kingmaAdamMethodStochastic2017b}.
While there are many other learning rate schedules, we focus on the use of the three aforementioned schedules within data streaming applications in this work.
For a comprehensive overview and detailed analysis on learning rate policies, we refer to \citet{wuDemystifyingLearningRate2019b}.

In contrast to the field of conventional batch learning, the impact of the learning rate in stream-based deep learning is a lesser studied issue. According to \citet{bifetMOAMassiveOnline2010} a machine learning model operating in such an environment must be able to
\begin{center}
	\begin{varwidth}{0.5\textwidth}
		\begin{description}
			\item[R1:] process a single instance at a time,\label{rq:single_instance}
			\item[R2:] process each instance in a limited amount of time,\label{rq:limited_time}
			\item[R3:] use a limited amount of memory,\label{rq:limited_memory}
			\item[R4:] predict at any time,\label{rq:predict_any_time}
			\item[R5:] adapt to changes in the data distribution.\label{rq:adapt_to_drift}
		\end{description}
	\end{varwidth}
\end{center}
These requirements lead to significant differences with respect to the problem of learning rate optimization compared to batch learning.
Nevertheless, only few studies on the impact and tuning of the learning rate in stream-based deep learning exist.

Under the requirements described above, however, we must reformulate Problem~\ref{eq:batch_lr_optim} as
\begin{equation}
	\label{eq:stream_lr_optim}
	\begin{alignedat}{3}
		& \!\min_{\eta_0, \ldots, \eta_T} & \, & \sum_{t=0}^{T}\mathcal{L}(y_t, f(X_t; \theta_{t-1})) &    &                               \\
		& \text{s.t.}                     &    & X_t, y_t  \sim p_t                           & \  & \forall t \in {1, \ldots, T}.
	\end{alignedat}
\end{equation}
Compared to Problem~\eqref{eq:batch_lr_optim}, the most apparent difference is that there is no separate validation data.
Instead, due to Requirement~\ref{rq:predict_any_time}, the goal is to minimize $\mathcal{L}$ with respect to the next instances $X_{t+1}, y_{t+1}$ at each timestep $t$.

This means that in contrast to Problem~\eqref{eq:batch_lr_optim}, not only the loss achieved by the final parameters $\theta_T$, but the loss suffered at every timestep of the stream contributes equally to the objective.
Therefore, speed of convergence is of much larger importance in the streaming setting, whereas the performance of the final parameters $\theta_T$ has relatively little impact.
Since memory is limited (Requirement~\ref{rq:limited_memory}), it is also not possible to continue training on previously observed data as long as $\mathcal{L}$ decreases, which puts an even greater emphasis on quick adaptation.
At the same time, a larger learning rate causing temporary loss increases, due to skipping over sharp minima can be suboptimal with respect to Problem~\ref{eq:stream_lr_optim} even if it eventually yields a lower loss.
Another difference to conventional batch learning is that the distribution $p_t$ of the data stream might, and in practice most likely will, be subjected to change in the form of concept drift over time.
Under such circumstances, the optimal parameter values $\theta^*$ move throughout the progression of the stream increasing the distance to the model parameters.


Since the theoretically optimal learning rate $\eta^*$ is proportional to the quadratic distance between initial and optimal parameters $||\theta_1 - \theta^*||^2$ (albeit under some constraints like the absence of noise)~\cite{carmonMakingSGDParameterFree2023}, it should be larger when concept drift occurs.

Based on this notion, \citet{kunchevaAdaptiveLearningRate2008} introduced an adaptive schedule that updates the learning rate using
\begin{equation}
	\eta_{t+1} = \eta_t^{1+(\mathcal{L}_t - \bar{\mathcal{L}}_{t-1})},
\end{equation}
where $\mathcal{L}_t$ represents the loss for the current sample, and $\bar{\mathcal{L}}_{t-1}$ a rolling mean of past losses.
By doing so, \citet{kunchevaAdaptiveLearningRate2008} increases in loss lead to increases in the learning rate and vice versa.
While this approach seems intuitively sound, it bears a high risk of $\eta$ increasing indefinitely, since increases in loss caused by an excessive learning rate would lead to a feedback loop.
Furthermore, strongly
% ^^ Does not make sense since direction of adaptations will reverse if $\eta$ increases above 1

We therefore propose a simple adaptation to decaying learning rate schedules that operates in a fixed value range and restore the model's ability to adapt to changes when needed by resetting $\eta$ to its original value if a concept drift has occurred.
To this end we apply an ADWIN drift detector~\cite{bifetLearningTimeChangingData2007} to the underlying model's prequential loss to reduce the drift detection task to a computationally less expensive univariate problem.
While our approach does not allow for a continuous adaptation like the technique by \citet{kunchevaAdaptiveLearningRate2008} and might therefore be less suited for more subtle drift, it offers more reliability since the initial learning rate cannot be exceeded.
\todo{Talk about drift-probability dependent resetting with P-KSWIN?}

Concept drift also complicates the tuning of $\eta$, since even if data is available beforehand drift would eventually cause the stream to diverge from the distribution of data used for tuning.
% In our experiments we investigate to which extent this effect impairs learning rate tuning on data available prior to deployment of a stream-based neural net. 
This effect, combined with the previously described differences in the evaluation scheme can cause conventional learning rate tuning to produce unsuitable results for stream-based learning.
% Alternatively: 
% Furthermore, in conventional batch learning, the standard practice for learning rate tuning is to optimize for the performance on the validation data~\cite{defazioLearningRateFreeLearningDAdaptation2023a}.
% As a result, this approach disregards the performance of the model throughout the tuning process and only selects the learning rate with the best final performance potentially yielding a value unsuitable for stream-based learning.
We therefore propose a slightly different online learning specific tuning approach, that aims to approximately solve Problem~\ref{eq:stream_lr_optim}.

To emulate
% a static version of 
the targeted data stream we continually draw samples with replacement from the tuning data in a bootstrapping procedure instead of training on all data for multiple epochs.
By doing so we aim to increase data variability% better: randomness?
, and therefore the resemblance to an actual data stream with random distributional shifts.
We then optimize $\eta$ with respect to the mean prequential performance over the emulated stream instead of the performance on a validation set.
For this purpose we use a basic grid-search as is customary in batch learning~\cite{defazioLearningRateFreeLearningDAdaptation2023a}.
We provide a detailed experimental evaluation of our approach in Section~\ref{sec:experiments}.


\section{Adaptive Optimizers}

While determining the learning rate through a separate tuning phase with parameter searches like grid- or random-search is still the de facto standard in deep learning~\cite{defazioLearningRateFreeLearningDAdaptation2023a}, this approach causes significant computational overhead.

To reduce this overhead, several previous works have developed \textit{adaptive optimizers}, which adjust the learning rate based on additional information about the loss landscape obtained from previous gradients at each optimization step, increasing the robustness with respect to the step size~\cite{duchiAdaptiveSubgradientMethods2011}.

One of the earlier optimizers in this category is \textit{AdaGrad}~\cite{duchiAdaptiveSubgradientMethods2011}, which divides the learning rate by the square root of the uncentered total sum of squares over all previous gradients, for each model parameter resulting in a parameter specific learning rate.
Unlike a single global value, parameter specific learning rates therefore not only influence the length, but also the direction of update steps, in case of AdaGrad by shifting updates in the direction of smaller gradients~\cite{wuWNGradLearnLearning2020}. % Weglassen? 
Among several other approaches like AdaDelta~\cite[see e.g.]{zeilerADADELTAAdaptiveLearning2012a} and RMSProp~\cite{tielemanLecture5rmspropDivide2012}, \citet{kingmaAdamMethodStochastic2017b} subsequently introduced Adam as an extension of AdaGrad, that additionally takes a momentum term of past gradients into account~\cite[see]{sutskeverImportanceInitializationMomentum2013} to speed up the convergence for parameters with consistent gradients.

While adaptive approaches such as AdaGrad and Adam have been shown to reduce the dependence on the learning rate, they often times still require manual tuning~\cite{wuWNGradLearnLearning2020}.
A problem that parameter-free variants of SGD aim to solve by estimating the optimal step size online as training progresses, thus eliminating the learning rate altogether.

For instance, \citet{schaulNoMorePesky2013} proposed \textit{vSGD}, which, like Adam, uses first and second order moments of the gradients as well as local curvature information to estimate $\eta$~\cite{schaulNoMorePesky2013}.
The authors obtain the latter by estimating positive diagonal entries of the Hessian with respect to the parameters through a back-propagation formula~\cite{schaulNoMorePesky2013}.
Even though \citet{schaulNoMorePesky2013} demonstrate \textit{vSGD's} robustness to non-stationary data distributions, it has, to the best of our knowledge, not been widely adopted in the online learning space.
Due to the lack of publicly available implementations of the non-trivial algorithm, we have not been able to evaluate vSGD at the time of writing. \todo{Did not evaluate vSGD since its old and there is no implementation. Is that valid?}

Instead of using curvature information for adapting $\eta$, the \textit{COCOB} algorithm proposed by~\citet{orabonaTrainingDeepNetworks2017} models parameter optimization as a gambling problem, in which the goal is to maximize the rewards obtained from betting on each gradient.
The model parameters are then computed based on the rewards accumulated over all previous timesteps~\cite{orabonaTrainingDeepNetworks2017}.
% Intuitively, this yields the expectation of the optimal value over an infinite number of models operating with different learning rates. <= Add this? \todo{Source: https://discuss.pytorch.org/t/any-cocob-optimizer-implementation-in-pytorch/3780/7}

-SGDHD: Meta learning approach optimizing the learning rate itself also using SGD
-WNGrad: estimates Lipschitz constant of gradients to derive learning rate $\eta$
-Mechanic: can wrap around any first order algorithm, removing the need of tuning $\eta$, uses a base online convex optimization (OCO) algorithm as well as a meta OCO to optimize the learning rate with respect to the theoretical upper convergence bound of SGD
-DoG: also optimizes theoretical upper bound by estimating $||\theta_0 - \theta^*||$ as $\max_{i<t}||\theta_0 - \theta_i||$
-DAdapt: modification of the AdaGrad step size applied to weighted dual averaging \cite{duchiDualAveragingDistributed2012}, together with our key innovation: D lower bounding


Furthermore, several studies developed paramater-free optimizers for specific areas of application such as time series forecasting~\cite{miyaguchiCograConceptDriftAwareStochastic2019,fekriDeepLearningLoad2021, zhangPOLAOnlineTime2021a}, federated learning~\cite{canonacoAdaptiveFederatedLearning2021} and recommender systems~\cite{ferreirajoseADADRIFTAdaptiveLearning2020}.
Due to our focus for the present work being general data stream applications, we did not further investigate these techniques either.

Despite the fact that this family of optimizers is inherently well-suited for the highly non-stationary streaming data~\cite{schaulNoMorePesky2013}, their application on this kind of data has rarely been investigated.
Besides \citet{schaulNoMorePesky2013} none of these optimizers target online learning scenarios and therefore do not explicitly consider concept drift, which raises the question of whether they are suitable for stream-based learning.
Previous work on parameter-free optimization of online deep learning models is rather limited, and mostly focused on specific applications.

\begin{table}[ht]
	\centering
	\small
	\begin{tabular}{@{}lllcc@{}}
		\toprule
		Optimizer & Runtime                         & Space             & Param. specific & LR free \\ \midrule
		DAdapt    & $\mathcal{O}(6D)$               & $\mathcal{O}(2D)$ & \xmark          & \cmark  \\
		DoG       & $\mathcal{O}(5D)$               & $\mathcal{O}(1D)$ & \xmark          & \cmark  \\
		Mechanic  & $\mathcal{O}(10D)$              & $\mathcal{O}(1D)$ & \xmark          & \cmark  \\
		WNGrad    & $\mathcal{O}(2D)$               & $\mathcal{O}(0)$  & \xmark          & \cmark  \\
		SGDHD     & $\mathcal{O}(2D)$               & $\mathcal{O}(1D)$ & \xmark          & \cmark  \\
		COCOB     & $\mathcal{O}(14D)$              & $\mathcal{O}(4D)$ & \cmark          & \cmark  \\
		Adam      & $\mathcal{O}(12D)$              & $\mathcal{O}(2D)$ & \cmark          & \xmark  \\
		vSGD      & $\mathcal{O}(21D)$\footnotemark & $\mathcal{O}(4D)$ & \cmark          & \cmark  \\ % Remove since not evaluated?
		AdaGrad   & $\mathcal{O}(5D)$               & $\mathcal{O}(1D)$ & \cmark          & \xmark  \\ \bottomrule
	\end{tabular}
	\caption{Overview of additional time- and space-complexity of adaptive first-order optimizers compared to basic SGD. Values are given in big O notation with respect to the number of model parameters $D$. We do not list convergence guarantees because the guarantees given in the original papers of different optimizers are based on different assumptions and are rarely compatible with streaming applications.}\label{tab:param_free_optims}
\end{table}
\footnotetext[1]{Complexity for feed-forward neural networks. Since \textit{vSGD} requires additional backpropagation steps, its complexity is architecture dependent.}


\section{Experiments}\label{sec:experiments}

\begin{table}[ht]
	\small
	\begin{tabular}{@{}clcccc@{}}
		\toprule
		Type                    & Data Stream            & Samples & Features & Classes \\
		\midrule
		\multirow{2}{*}{Synth.} & RBF abrupt             & 20000   & 20       & 5       \\
		                        & RBF incremental        & 20000   & 20       & 5       \\
		\midrule
		\multirow{5}{*}{Real}   & Insects abrupt         & 52848   & 33       & 6       \\
		                        & Insects incremental    & 57018   & 33       & 6       \\
		                        & Insects incr.-grad.    & 24150   & 33       & 6       \\
		                        & Covertype\footnotemark & 100000  & 54       & 7       \\
		                        & Electricity            & 45312   & 8        & 2       \\
		\bottomrule
	\end{tabular}\label{tab:datasets}
	\caption{Datasets used for experimental evaluations.}
\end{table}

\footnotetext[2]{We used the first 100k from a total of 581k examples only.}
% \section{Results}


% \begin{table*}[t]
%     \centering
%     \small
%     \begin{tabular}{lllllll}
%         \toprule
%         Optimizer                & Schedule     & RBF abrupt                     & RBF incr.                      & Covertype                      & Insects abrupt                 & Electricity                    \\
%         \midrule
%         \multirow[c]{7}{*}{SGD}  & Fixed        & 94.79±.32                      & 70.95±2.89                     & 83.42±.50                      & 71.50±.08                      & 73.77±.40                      \\
%                                  & Exp.         & \underline{94.86±.36}          & 72.79±2.35                     & \underline{83.56±.38}          & 71.87±.38                      & 73.90±.55                      \\
%                                  & Exp. Reset   & \underline{94.86±.20}          & 73.77±.98                      & 83.28±.43                      & 71.75±.22                      & 73.85±.64                      \\
%                                  & Step         & \underline{94.94±.24}          & 72.51±1.70                     & \bfseries\underline{83.61±.18} & 71.96±.20                      & 73.57±.89                      \\
%                                  & Step Reset   & \bfseries\underline{94.99±.17} & 72.41±1.75                     & \underline{83.47±.38}          & 71.62±.18                      & 73.70±.88                      \\
%                                  & Cyclic       & 94.79±.32                      & \bfseries\underline{74.96±.86} & \underline{83.44±.08}          & 71.74±.39                      & 68.38±.81                      \\
%                                  & Cyclic Reset & 94.79±.32                      & \underline{74.45±1.69}         & 83.42±.50                      & 71.69±.25                      & 68.52±1.02                     \\ \midrule
%         \multirow[c]{3}{*}{Adam} & Fixed        & 93.45±.30                      & 69.26±5.14                     & 79.01±.27                      & 75.38±.24                      & 69.79±.54                      \\
%                                  & Exp.         & 92.93±.63                      & 64.72±5.35                     & 79.04±.53                      & \bfseries\underline{75.61±.08} & 69.85±.46                      \\
%                                  & Exp. Reset   & 94.03±.48                      & 63.94±2.05                     & 77.27±.48                      & 72.65±.84                      & 69.12±.94                      \\
%         \midrule
%         AdaGrad                  & Fixed        & 92.45±1.37                     & 52.87±6.62                     & 81.68±.35                      & 74.87±.40                      & 76.99±1.20                     \\
%         SGDHD                    & Fixed        & 93.92±.31                      & 72.29±2.90                     & 83.36±.25                      & 70.67±.06                      & 73.83±.32                      \\
%         COCOB                    & Fixed        & 93.40±.38                      & 63.52±2.70                     & 82.27±.46                      & 74.75±.11                      & \bfseries\underline{84.30±.56} \\
%         WNGrad                   & Fixed        & 87.30±.68                      & 44.92±.73                      & 76.98±.15                      & 66.25±.19                      & 70.80±.59                      \\
%         DAdaptSGD                & Fixed        & 74.91±4.22                     & 45.47±2.75                     & 76.69±.79                      & 50.05±11.26                    & 66.03±1.75                     \\
%         DoG                      & Fixed        & 92.73±.59                      & 73.17±2.72                     & 83.07±.64                      & 70.59±.26                      & 71.53±.70                      \\
%         \bottomrule
%     \end{tabular}
%     \caption{Average prequential accuracy [\%] for best learning rate $\eta^*$.}\label{tab:top1_accuray}
% \end{table*}

\begin{table*}[t]
	\centering
	\small
	\begin{tabular}{lllllll}
		\toprule
		Optimizer                & Schedule     & RBF abrupt            & RBF incr.              & Covertype             & Insects abrupt       & Electricity          \\
		\midrule
		\multirow[c]{7}{*}{SGD}  & Fixed        & 93.70±.90             & 69.33±1.36             & 83.08±.18             & 71.12±.08            & 73.12±.42            \\
		                         & Exp.         & 94.00±.52             & 69.24±1.06             & 82.65±.19             & 71.32±.19            & 73.06±.42            \\
		                         & Exp. Reset   & 94.28±.37             & 69.76±.92              & 82.70±.27             & 71.27±.14            & 73.05±.45            \\
		                         & Step         & 94.00±.68             & 69.16±.99              & 82.74±.10             & 71.39±.17            & 72.96±.48            \\
		                         & Step Reset   & 94.04±.70             & 69.75±.81              & 83.03±.13             & 71.19±.14            & 73.18±.50            \\
		                         & Cyclic       & \underline{94.45±.25} & \underline{73.72±1.16} & \bfseries{83.40±.21}  & 71.41±.20            & 67.80±1.03           \\
		                         & Cyclic Reset & \bfseries{94.50±.21}  & \bfseries{73.78±1.15}  & \underline{83.33±.13} & 71.39±.15            & 67.83±1.00           \\ \midrule
		\multirow[c]{3}{*}{Adam} & Fixed        & 92.77±.41             & 66.46±4.39             & 78.85±.22             & \bfseries{75.08±.13} & 69.23±.41            \\
		                         & Exp.         & 92.17±.82             & 63.91±3.52             & 78.53±.27             & 74.88±.15            & 69.33±.40            \\
		                         & Exp. Reset   & 92.83±1.19            & 62.03±2.59             & 77.05±.08             & 72.04±.49            & 68.31±.40            \\
		\midrule
		AdaGrad                  & Fixed        & 91.34±.83             & 50.39±3.60             & 81.07±.22             & 74.31±.34            & 76.64±1.92           \\
		SGDHD                    & Fixed        & 91.03±.45             & 63.38±1.55             & 82.33±.12             & 67.35±.16            & 73.10±.10            \\
		COCOB                    & Fixed        & 93.40±.38             & 63.52±2.70             & 82.27±.46             & 74.75±.11            & \bfseries{84.30±.56} \\
		WNGrad                   & Fixed        & 87.23±1.24            & 44.79±.76              & 76.95±.15             & 66.14±.15            & 70.74±.59            \\
		DAdaptSGD                & Fixed        & 74.91±4.22            & 45.47±2.75             & 76.69±.79             & 50.05±11.26          & 66.03±1.75           \\
		DoG                      & Fixed        & 92.73±.59             & \underline{73.17±2.72} & 83.07±.64             & 70.59±.26            & 71.53±.70            \\
		\bottomrule
	\end{tabular}
	\caption{Average prequential accuracy [\%] for the three best learning rates.}\label{tab:top3_accuray}
\end{table*}

\begin{table*}
	\small
	\centering
	\begin{tabular}{llllll}
		\toprule
		Resetting Approach & RBF abrupt            & RBF incr.              & Covertype           & Insects abrupt        & Electricity           \\
		\midrule
		ADWIN (Two Tailed) & \bfseries{94.28±.37}  & \underline{69.76±.92}  & 82.70±.27           & \underline{71.27±.14} & \underline{73.05±.45} \\
		ADWIN (One Tailed) & \underline{94.25±.38} & \underline{70.22±2.58} & 82.64±.20           & \underline{71.25±.15} & \underline{73.00±.50} \\
		ADWIN Weight Reset & 71.79±.73             & 65.07±.31              & 82.54±.16           & 50.97±.36             & 70.19±1.00            \\
		KSWIN              & \underline{94.23±.55} & \underline{70.10±1.98} & \bfseries 83.01±.06 & \bfseries 71.38±.16   & \underline{73.13±.31} \\
		P-KSWIN            & 93.86±.48             & \bfseries 70.71±1.38   & \bfseries 83.01±.18 & \underline{71.25±.15} & \bfseries 73.26±.43   \\
		\bottomrule
	\end{tabular}
	\caption{Average prequential accuracy [\%] for the three best learning rates.}
\end{table*}



\begin{figure}
	\centering
	\includegraphics[width=.45\textwidth]{figures/pretune_1x64_acc_lr_exp_schedule.pdf}
	\caption{Pre-tuned LR (LR that maximizes accuracy on pre-tuning data) and resulting accuracy on data streams when using SGD and an exponential learning rate schedule with 500 or 1000 separate tuning samples. Results are averaged over all real-world datasets. The shaded area represents the 1$\sigma$-interval.}\label{fig:pretune_lr_accuracy}
\end{figure}

\begin{figure}
	\centering
	\includegraphics[width=.47\textwidth]{figures/pretune_1x64_fixed_vs_exp_schedule.pdf}
	\caption{Accuracy achieved by pre-tuning on 500 or 1000 samples when using SGD with a fixed LR schedule (top) or an exponential schedule (bottom), averaged over all real-world datasets. The shaded area represents the 1$\sigma$-interval.}\label{fig:pretune_fixed_vs_exp_lr}
\end{figure}

\begin{figure}
	\centering
	\includegraphics[width=0.47\textwidth]{figures/lr_norms_optims_insects_abrupt.pdf}
	\caption{Prequential accuracy, binary cross-entropy loss and LR norms $||\eta_t||$ over time for various optimization algorithms on Insects abrupt. Each dashed vertical line represents a concept drift. Lines are exponentially smoothed with a factor of 0.8.}
\end{figure}
\begin{figure}
	\centering
	\includegraphics[width=.47\textwidth]{figures/batch_size_lr_wstd.pdf}
	\caption{Average accuracy over static and incrementally drifting synthetic data streams in relation to SGD mini-batch size and learning rate $\eta$. Shaded areas mark the 1$\sigma$ interval.}\label{fig:batch_size_lr}
\end{figure}

\begin{figure}
	\centering
	\includegraphics[width=.47\textwidth]{figures/accuracy_vs_time.pdf}
	\caption{Prequential accuracy for different batch sizes on RBF abrupt data stream.}
\end{figure}

We ran prequential evaluations using basic SGD with variable batch sizes and learning rates for synthetic data streams with and without incremental concept drift, the results of which are displayed in \Cref{fig:batch_size_lr}. For static data, the average prequential accuracy over the entire stream gradually improves when moving up from an inadequately low learning rate until a certain point where training begins to diverge and performance consequently crashes. Based on our results, there seems to be an inverse relationship between batch size and both the optimal learning rate and the optimal accuracy, with larger batch sizes seemingly increasing the risk of divergence.

% The primary cause of this relationship can be seen in \Cref{fig:trajectory_batch_sizes_0.5_lr}, which depicts the trajectories of parameters of two identical MLPs trained on a bivariate data stream using batch sizes of either 4 or 16 samples. With a batch size of 16, the model has to “wait” four times longer until it receives new gradient information than the one trained with a batch size of 4. As a result, the gradient information gained from each sample on average becomes less recent and therefore less accurate with a larger batch size. Example: in \Cref{fig:trajectory_batch_sizes_0.5_lr}

\begin{equation}
\end{equation}


This effect is much stronger in the presence of concept drift as the results for RBF Incremental show.

<-could be explained by the fact that the presence of concept drift exacerbates the gradient stochasticity caused by the delay between observation and learning of samples.


\begin{figure}[ht]
	\centering
	\begin{tikzpicture}
		% Upper image
		\node[inner sep=0pt] (upper) {\includegraphics[width=0.4\textwidth]{figures/sgd_trajectory_optims1.pdf}};

		% Lower image
		\node[inner sep=0pt, below=3mm of upper] (lower){\includegraphics[width=0.4\textwidth]{figures/sgd_trajectory_optims2.pdf}};

		\path ([xshift=-8pt]upper) -- ([xshift=-8pt]lower) node[midway] (text){Concept Drift};
		\draw[->] ([xshift=-5pt, yshift=5pt]upper.south-|text.west) -- ([xshift=-5pt, yshift=-3pt]lower.north-|text.west);
		\draw[->] ([xshift=5pt, yshift=5pt]upper.south-|text.east) -- ([xshift=5pt, yshift=-3pt]lower.north-|text.east);

	\end{tikzpicture}
	\caption{Parameter trajectory of COCOB~\cite{orabonaTrainingDeepNetworks2017}, DoG~\cite{ivgiDoGSGDBest2023} and Adam~\cite{kingmaAdamMethodStochastic2017b} on synthetic data stream with abrupt concept drift. Marker colors depict the expected prequential loss over the last 16 data instances.}
\end{figure}

\section{Conclusion}


\bibliography{aaai24}

\end{document}
